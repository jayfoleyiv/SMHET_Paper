\documentclass[journal=jacsat,manuscript=article]{achemso}
\pdfoutput=1
\usepackage{gensymb}
\usepackage{amsmath}
\usepackage{graphicx}
\usepackage{epsfig}
\usepackage{multirow}
\usepackage{color,soul}


% Authors / affiliations
% Authors in alphabetical order except for corresponding author (JJF), need 
% to determine if this is the proper order and if we want to have co-first authors!

\author{Jason Codrington}
\affiliation{Department of Chemistry, William Paterson University, 300 Pompton Road, Wayne, NJ, 07470, USA}
\author{Noor Eldabagh}
\affiliation{Department of Chemistry, William Paterson University, 300 Pompton Road, Wayne, NJ, 07470, USA}
\author{Kimberly Fernando}
\affiliation{Department of Chemistry, William Paterson University, 300 Pompton Road, Wayne, NJ, 07470, USA}
\author{Jonathan J. Foley IV}
\affiliation{Department of Chemistry, William Paterson University, 300 Pompton Road, Wayne, NJ, 07470, USA}
\email{foleyj10@wpunj.edu}

%Title of paper
\title{Highly-tunable hot-electron distributions from scattering-mediated absorption}
% Date
\date{\today}


% Being document
\begin{document}

\begin{abstract}

\end{abstract}

\maketitle

%%\end{document}


\section{Introduction}

\section{Theoretical Model}
-Employ a time-domain multi-scale electrodynamics/quantum mechanics model 
	- Finite-difference time-domain method is used for the 

\subsection{The Finite-difference time-domain method}


\subsection{The Discrete Dipole Approximation}


\section{Results}


\section{Concluding Remarks}

\section{Methods}

\subsection{Finite-difference time-domain method}

\subsection{Time-dependent configuration-interaction singles method}
We use the time-dependent configuration interaction singles (TD-CIS) method to elucidate the
electronic dynamics in the metal nanoparticles driven by the electric fields resulting from 
dielectric scattering resonances, plasmon resonances, or freely propagating light.  
The TDCIS equations arise directly from the time-dependent Schr\"odinger equation (TDSE) when the wavefunction
is written as a superposition of a ground state reference configuration, $|\Phi_0\rangle$, and configurations which are singly excited relative
to $\Phi_0$,  In the CIS model, the wavefunction has the form
\begin{equation}
|\Psi_{CIS}\rangle = c_0 |\Phi_0 \rangle + \sum_{i,a} c_i^a |\Phi_i^a\rangle,
\end{equation}
where the configuration $|\Phi_i^a\rangle$ has an electron excited from orbital $i$ to orbital $a$, 
and $c_0$ and $c_i^a$ are complex expansion coeffients.  Unless otherwise specified, indices $i, j$ will indicate
orbitals which are occupied in the ground state reference configuration and indices $a, b$ will indicate orbitals
which are unoocupied in the ground state reference.

Each many-electron configuration is an anti-symmetrized product of one-electron orbitals which are chosen to be energy eigenstates of 
an appropriate one-electron Hamiltonian, further elaborated in subsequent sections. 
We neglect electron-electron repulsion in our many-electron Hamiltonian, so the ground state configuration
can be exactly described by the anti-symmetrized product of the first $N/2$ one-electron orbitals, 
$|\Phi_0\rangle = |\psi_1 ... \psi_i \psi_j ... \psi_{N/2} \rangle$, where $\psi_q$ denotes the $q^{th}$ one-electron orbital.  

The time-evolution of the wavefunction can be subsumed in the expansion coefficients, which allows the TDSE to be written 
\begin{equation}\label{TDCIS}
i\hbar \frac{ d}{dt} {\bf c}(t) = {\bf H}(t) {\bf c}(t)
\end{equation}
where ${\bf c}(t)$ is the vector of complex expansion coefficients and ${\bf H}(t)$ is the time-dependent Hamiltonian
matrix.  The Hamiltonian matrix is comprised of three unique blocks in the CIS model,  
\begin{equation}
  {\bf H}(t) 
  \mbox{=}
  \begin{pmatrix}
    \langle \Phi_0 | \hat{H}(t) | \Phi_0 \rangle    &     \langle \Phi_0 | \hat{H}(t) | \Phi_i^a \rangle    \\
  \langle \Phi_j^b | \hat{H}(t) | \Phi_0 \rangle    &   \langle \Phi_j^b | \hat{H}(t) | \Phi_i^a \rangle \end{pmatrix}
  \begin{pmatrix} t \\ 0  \end{pmatrix}.
\end{equation}
In our work, the Hamiltonian operator has a component for the electronic energy and a component for a dipolar interaction of the electrons with the electric
field,
\begin{equation}
\hat{H}(t) = \hat{H}_{el} - {\bf E}(t) \cdot \hat{\mu},
\end{equation}
and derives time-dependence through the time-dependence of the electromagnetic field that interacts
with the electrons described by the CIS wavefunction.  Both the electronic and dipolar contributions can be written 
as sums of one-electron terms %% Probably want to include slater rules... also mention that everything is 1-electron
The specific forms of the components of the Hamiltonian
operator are covered in more detail in the following sections. 

The multiplication of the Hamiltonian matrix on the coefficient vector generates the gradient of the coefficient vector in time, and 
a variety of algorithms are known that use this information to propagate the wavefunction in time.  Here we use a symplectic integrator
described in Ref.{\it Sanz-Serna's paper}.

\subsection{Electronic structure of metal nanospheres}
For spherical metal nanoparticles, we approximate the one-electron orbitals as energy eigenstates of the particle-in-a-spherical-well
Hamiltonian, 
\begin{equation}
\hat{H}_{PIS} = -\frac{\hbar^2}{2 m r^2} 
\left(\frac{\partial}{\partial r} \left(r^2 \frac{\partial}{\partial r}\right)
+ \left(\frac{1}{{\rm sin}^2 \theta} 
\left({\rm sin}\theta \frac{\partial}{\partial \theta}
\left({\rm sin}\theta \frac{\partial}{\partial \theta} \right)
+ \frac{\partial^2}{\partial \phi^2} \right) \right)
\right). 
\end{equation}
For a particle with radius $R$, the potential is 0 when $r<R$ and infinity with $r \geq R$. 
The potential is 0 
The energy eigenstates have the form 
\begin{equation}
\psi_{n,l,m}(r,\theta,\phi) = j_l(\alpha r) Y_{l,m}(\theta,\phi),
\end{equation}
where $j_l(\alpha r)$ are the spherical Bessel functions are 
$Y_{l,m}(\theta,\phi)$ are the spherical Harmonics.  
In our numerical implementation, we use the asymptotic approximation of the spherical Bessel functions 
$j_l(\alpha r) \approx {\rm cos}(\alpha r - \frac{\pi}{2} (l+1))/r$.  To evaluate the spherical harmonics, we use a recursive 
algorithm described in Ref. {\it Numerical Recipes in C} to evaluate the associated Legendre polynomials.  The energy eigenfunctions
must vanish at $r=R$, which leads to the following form for the one-electron orbitals:
\begin{equation}
\psi_{n,l,m}(r,\theta,\phi) = 
\frac{2}{\sqrt{R}} \frac{ {\rm cos}( \frac{\pi}{2} \frac{r}{R} - \frac{\pi}{2} (l+1)  }{r} Y_{l,m} (\theta,\phi).
\end{equation}
The energy eigenvalues have the form 
\begin{equation}
E_{n,l} = \frac{\hbar^2 \pi^2}{8 m R^2} \left(2 n + l + 2\right)^2.
\end{equation}
From the definition of the spherical polar coordinates, the $x$, $y$, and $z$ components of the transition dipole integrals have the form
\begin{align}
\mu_x = -e \int_0^R \int_0^{2\pi} \int_0^{\pi} \psi^*_{n,l,m} r {\rm sin}\theta {\rm cos}\phi \psi_{n',l',m'}  r^2 {\rm sin}\theta dr d\theta d\phi \\
\mu_y = -e \int_0^R \int_0^{2\pi} \int_0^{\pi} \psi^*_{n,l,m} r {\rm sin}\theta {\rm sin}\phi \psi_{n',l',m'}  r^2 {\rm sin}\theta dr d\theta d\phi \\
\mu_z = -e \int_0^R \int_0^{2\pi} \int_0^{\pi} \psi^*_{n,l,m} r {\rm cos}\theta               \psi_{n',l',m'}  r^2 {\rm sin}\theta dr d\theta d\phi.
\end{align}
In this work, we neglect the $x$ and $y$ components of the field and evaluate $mu_z$ numerically, where the integral can be
simplified to
\begin{equation}
\mu_z = \delta_{m,m'} \int_0^R \int_0^{\pi} \frac{2}{\sqrt{R}} \frac{{\rm cos( \frac{\pi}{2} \frac{r}{R} - \frac{\pi}{2} (l+1)  }{r} P_{l'}^{m'}({\rm cos}\theta)  {\rm cos}\theta               \frac{2}{\sqrt{R}} \frac{{\rm cos( \frac{\pi}{2} \frac{r}{R} - \frac{\pi}{2} (l+1)  }{r} P_{l'}^{m'}({\rm cos}\theta)  r^2 {\rm sin}\theta dr d\theta 
\end{equation}
\subsection{Electronic structure of metal nanocubes}

\section{Acknowledgement}

\bibliography{SMHET} 

\end{document}

