\documentclass[journal=jpclcd,manuscript=article]{achemso}
\pdfoutput=1
\usepackage{gensymb}
\usepackage{amsmath}
\usepackage{graphicx}
\usepackage{epsfig}
\usepackage{multirow}
\usepackage{multicol}


% Authors / affiliations
% Authors in alphabetical order except for corresponding author (JJF), need 
% to determine if this is the proper order and if we want to have co-first authors!

\author{Jason Codrington$^{\dagger}$}
\affiliation{Department of Chemistry, William Paterson University, 300 Pompton Road, Wayne, NJ, 07470, USA}
\author{Noor Eldabagh$^{\dagger}$}
\affiliation{Department of Chemistry, William Paterson University, 300 Pompton Road, Wayne, NJ, 07470, USA}
\author{Kimberly Fernando$^{\dagger}$}
\affiliation{Department of Chemistry, William Paterson University, 300 Pompton Road, Wayne, NJ, 07470, USA}
\author{Jonathan J. Foley IV}
\affiliation{Department of Chemistry, William Paterson University, 300 Pompton Road, Wayne, NJ, 07470, USA}
\email{foleyj10@wpunj.edu}

%Title of paper
\title{Unique hot carrier distributions from scattering mediated absorption}
% Date
\date{\today}


% Being document
\begin{document}

\begin{tocentry}
\includegraphics[width=9cm]{figs/nanosphere_WGMv2.png}
\end{tocentry}

\begin{abstract}

Light-initiated generation of energetic carriers has attracted considerable attention as a paradigm for 
photocatalysis and solar energy conversion, and the use of noble metal nanoparticles that support localized surface
plasmon resonances has been widely explored as a medium for realizing this paradigm.  It was recently
shown that composite nanostructures that enable the interplay between dielectric scattering resonances and broad-band
absorption in small metal nanostructures, a phenomenon termed scattering mediated absorption, can be 
used to mediate energetic carrier transfer and selective photochemistry with 
low-intensity light while completely circumventing plasmon resonance.  In this work, we develop 
a multi-scale modeling approach for elucidating the hot carrier dynamics initiated by scattering mediated
absorption.  Our calculations reveal that unique hot carrier distributions and dynamics arise 
from scattering mediated absorption as compared to plasmon excitation, and also suggest that in 
a variety of circumstances, scattering mediated absorption may lead to more
efficient hot carrier generation than plasmon resonance under the same external illumination
conditions.  These results are an important first step in understanding the phenomena of
scattering mediated hot carrier generation, which has potential for expanding the
palette of materials that can be utilized for hot carrier mediated photochemistry beyond plasmonic metals,
and for enabling unique pathways for photocatalytic transformations.
\end{abstract}

%%\maketitle

%%\end{document}

\section{Introduction}

Various strategies that exploit the optical properties of metal nanoparticles, namely their ability to support localized surface plasmons resonances [LSPR], 
which are collective oscillations of their 
conduction electrons driven by light, have been explored recently with the aim of using low-intensity light to efficiently drive chemical 
reactions~\cite{LCI_NatureMater_2011,KAC_ACSCatalysis_2013,ZLQ_RSCAdvances_2015,PKL_AccChemRes_2015}.  The interest in this area has 
been motivated partly by the fact that metal nanoparticles (most prominently silver and gold) are exceptionally good absorbers of visible light, 
so they are ideal candidates for harvesting solar photons~\cite{AP_NatMat_2010}.  
Importantly, the resonant properties of plasmonic particles are highly tunable by parameters under synthetic control such as geometry, composition, surface chemistry, and the surrounding 
environment~\cite{SX_Science_2002,BCN_ChemRev_2005,GB_NatPhoton_2010}.  An emerging paradigm that exploits LSPR for photocatalysis is known as Plasmon-mediated Hot-Electron Transfer [PHET],
and a growing number of reports are demonstrating the ability of PHET to catalyze energetically demanding chemical 
reactions~\cite{CXL_NatureChem_2011,MZL_Science_2013,MLL_NanoLett_2013,LFP_AC_2015,ZHX_NatPhoton_2016,ZJM_ACSNano_2016,SZZ_PNAS_2016,SCR_JPCC_2016}.

In PHET, the collective plasmon excitation decays rapidly (on a ~10 femtosecond timescale) to a non-equilibrium distribution of energetic electron-hole pairs, or a so-called hot carrier 
distribution~\cite{KAC_ACSCatalysis_2013,GZG_JPCC_2013,SNJ_NatComm_2014,WCM_Science_2015,MWW_NatComm_2015, BSN_ACSNano_2016}.  
Hot carriers can can deposit energy into reactive degrees of freedom of molecules adsorbed to the nanoparticle surface, thereby initiating chemical transformations.  Despite the considerable progress made in 
demonstrating the potential of the paradigm of PHET, its widespread application faces several challenges. The intrinsic optical properties of noble metals that give rise to the extraordinarily 
large absorption cross sections associated with LSPR are also fundamentally related to the broad energy spectra and short lifetimes associated with LSPR and the subsequent hot carrier 
distributions~\cite{KS_JCP_1983}. 
Furthermore, the timescale of hot carrier relaxation competes 
with transfer to adsorbate states (both occur on 100 fs timescales), which fundamentally limits the efficiency of energy transfer~\cite{WCM_Science_2015}.  Finally, the most promising plasmonic materials typically have poor catalytic activity, and similarly,
good catalytic metals typically are poor light harvesters~\cite{SZZ_PNAS_2016}.

Considering this, an incredible opportunity exists to identify new classes of structures for mediating light-matter interactions and energy transfer events that offer the same advantages as metal nanoparticles, 
namely exceptional light-harvesting potential, while also offering greater selectivity and efficiency in energy transfer, as well as tunable surface chemistry for enhanced catalytic activity.  
Ideally, such structures could be made mostly, if not entirely, from cost-effective materials.  Recent progress towards this aim has been made by designing hybrid 
nanostructures that effectively delegate the light-harvesting and catalytic functions to separate components of the structure.  Recently, Halas and co-workers 
demonstrated an antenna-reactor concept that leverages the near-field enhancement from aluminum plasmons to generate energetic carriers on 
palladium islands and showed the efficacy of this strategy for the photcatalytic transformation of acetylene to ethylene~\cite{SZZ_PNAS_2016}.
Two of the current authors, along with Sun and co-workers, demonstrated a phenomena known as scattering mediated absorption [SMA] where dielectric
scattering resonances in SiO$_2$ nanospheres were utilized to induce resonant absorption in non-plasmonic platinum nanoparticles in SiO$_2$/Pt nanohybrids~\cite{ZHX_NatPhoton_2016}.
The SMA phenomena was also observed to induce highly selective photocatalytic oxidation of benzyl alcohol to benzaldehyde~\cite{ZHX_NatPhoton_2016}.
Zhang {\it et al.} independently described a SMA phenomenon in Au-TiO$_2$ nanohybrids that leveraged so-called whispering gallery modes to 
enhance plasmonic and non-plasmonic absorption in gold nanoparticles, and demonstrated these structure's efficacy for photocatalytic water splitting~\cite{ZJM_ACSNano_2016}.
Interestingly, hot carrier transfer was implicated in the photocatalytic mechanisms that resulted from these SMA phenomena~\cite{ZHX_NatPhoton_2016,ZJM_ACSNano_2016}.
The prospect of using SMA in hybrid dielectric/metal nanostructures to initiate hot carrier generation and transfer is particularly 
compelling as it could completely circumvent plasmon excitation.  Not only does this open up possibilities to utilize a broad palette of earth-abundant and cost-effective materials, it also
presents the possibility of realizing unique photocatalytic pathways owing to differences in the distributions and dynamics of energetic carriers produced by SMA compared
to those produced by plasmon excitation.    

The push to identify novel structures for mediating hot carrier generation and transfer has been paralleled by 
efforts to develop theoretical methodologies to elucidate these processes.  The underlying electronic
structure has been treated both within free-electron models confined by potential wells~\cite{GZG_JPCC_2013,ZG_JPCC_2014,MLK_ACSNano_2014,KPB_SciRep_2015,SAG_ACSPhotonics_2016} (here called
``particle-in-a-well" [PIW] models), as well as by {\it ab initio} approaches~\cite{SNJ_NatComm_2014,BMN_NatComm_2015,MWW_NatComm_2015,BSN_ACSNano_2016}.
Using PIW models, Govorov and co-workers developed a theory of 
hot carrier generation within the framework of time-dependent perturbation theory that
has elucidated a variety of shape- and size-dependent factors for optimizing 
the hot carrier generation~\cite{GZG_JPCC_2013,ZG_JPCC_2014}.  A similar approach was also pursued by Kumarasinghe {\it et al.} suggesting
that nanorods are exceptionally good structures for hot carrier generation~\cite{KPB_SciRep_2015}.  Garc\'ia de Abajo and co-workers recently described a 
quantum master equation approach with an underlying PIW model that elucidated a number of key factors that influence
hot carrier excitation and decay dynamics~\cite{SAG_ACSPhotonics_2016}.  The utilization of {\it ab initio} approaches by Sundararaman {\it et al} 
has also provided valuable insights into the role that a metal's band-structure plays
in determining the efficiency of hot carrier generation, and particularly in the asymmetry between hot-electron
and hot-hole generation~\cite{SNJ_NatComm_2014}.  Nordlander and co-workers have directly compared free-electron and {\it ab initio} approaches
and found negligible impact on hot carrier generation and dynamics in Ag nanospheres resulting from electron correlation~\cite{MLK_ACSNano_2014}, 
though Bernardi {\it et al} have demonstrated that many-body effects are important in the hot carrier dynamics resulting
from surface plasmon polaritons in gold and silver~\cite{BMN_NatComm_2015}.

We develop a novel approach for studying hot carrier dynamics that may arise from arbitrary electromagnetic
fields, including the unique near-fields that arise from SMA on hybrid dielectric-metal nanoparticles and LSPR
on noble metal nanoparticles.  Importantly, this method goes beyond a perturbative inclusion of the electric field and enables us to follow the electronic degrees of freedom subject to strong fields with arbitrary time-dependence, which are characteristic of SMA and LSPR.  

We consider the electronic degrees of freedom on the metal nanoparticle subject to the time-dependent Hamiltonian 
\begin{equation}\label{eq:TDHam}
\hat{H}(t) = \hat{H}_{el} - {\bf E}(t) \cdot \hat{\mu}, 
\end{equation}
where the specific form of ${\bf E}(t)$ derives from a rigorous time-domain electrodynamics calculation with a realistic model
of the nanostructures in question and $\hat{H}_{el}$ describes non-interacting electrons confined to metal nanocubes [NCs] by an infinite potential well.  
We compute the time-domain field 
using a commercial simulator based on the finite-difference time-domain [FDTD] method~\cite{Lumerical}.  
Incorporating  ${\bf E}(t)$ from these FDTD simulations into the time-dependent Hamiltonian enables us to investigate the unique ways
in which the electronic degrees of freedom evolve under the influence of the distinct time-varying fields that result
from LSPR and from dielectric scattering resonances, allowing us a unique window into the hot carrier dynamics resulting from LSPR as compared
to SMA.  Further details on the FDTD simulations can be found in the Supplementary Information [SI]. 

The many-electron wavefunction of the metal nanoparticles is expanded in terms of a configuration-interaction expansion that
includes all singly-excited configurations,
\begin{equation}\label{eq:CIS}
|\Psi_{CIS}\rangle = c_0 |\Phi_0 \rangle + \sum_{i,a} c_i^a |\Phi_i^a\rangle,
\end{equation}
where the configuration $|\Phi_i^a\rangle$ has an electron excited from orbital $i$ to orbital $a$, 
and $c_0$ and $c_i^a$ are complex expansion coefficients.  Unless otherwise specified, indices $i, j$ will indicate
orbitals which are occupied in the ground state reference configuration and indices $a, b$ will indicate orbitals
which are unoccupied in the ground state reference.  

The time-evolution of the wavefunction can be subsumed in the expansion coefficients, which allows the TDSE to be written 
\begin{equation}\label{TDCIS}
i\hbar \frac{ d}{dt} {\bf c}(t) = {\bf H}(t) {\bf c}(t)
\end{equation}
where ${\bf c}(t)$ is the vector of complex expansion coefficients and ${\bf H}(t)$ is the time-dependent Hamiltonian
matrix.  The Hamiltonian matrix is comprised of three unique classes of elements in the CIS model,  
\begin{equation}
  {\bf H}(t) 
  \mbox{=}
  \begin{pmatrix}
    \langle \Phi_0 | \hat{H}(t) | \Phi_0 \rangle    &     \langle \Phi_0 | \hat{H}(t) | \Phi_i^a \rangle    \\
  \langle \Phi_j^b | \hat{H}(t) | \Phi_0 \rangle    &   \langle \Phi_j^b | \hat{H}(t) | \Phi_i^a \rangle \end{pmatrix}.
\end{equation}
Explicit expressions for each class of matrix elements are given in the SI.

Given the simplicity of the underlying electronic Hamiltonian, the field-free Hamiltonian matrix is diagonal, and only the dipolar
interaction of the nanoparticle with the external field can induce transitions among the electronic configurations, hence the treatment in 
this work neglects excited-state decay contributions from electron-electron scattering.  These contributions will be explored in future work.  
Also because of the diagonal nature of the field-free Hamiltonian, each configuration $|\Phi_i^a\rangle$ is an eigenfunction
of the field-free Hamiltonian.  This simplifies the interpretation of the electronic structure relative to the CIS 
wavefunction in quantum chemistry applications where electron repulsion is included in the Hamiltonian
and the excited electronic eigenfunctions are linear combinations of singly-excited configurations. 
The multiplication of the Hamiltonian matrix on the coefficient vector generates the gradient of the coefficient vector in time, and
a variety of algorithms have been developed to use this information to propagate the wavefunction in time.  Here we use a symplectic integrator
described in Ref.~\citenum{SP_JCP_96}.  Propagation of the CIS wavefunction is referred to as the TDCIS method throughout.

We analyze the hot carrier distribution and dynamics that results from SMA and LSPR excitation by computing the 
instantaneous populations of orbitals both above and below the Fermi level of the metal nanostructure.   
In our model, the orbitals are energy eigenstates of a 1-electron Hamiltonian and have well-defined kinetic energy,
and the orbital populations are given by the diagonal elements of the 1-electron reduced density matrix [1-RDM],
\begin{equation}
^1D^q_q(t) = \langle \Psi(t) | \hat{a}^{\dagger}_q \hat{a}_q | \Psi(t) \rangle,
\end{equation} 
where the second-quantized operator $\hat{a}_q^{\dagger}$ ($\hat{a}_q$) creates (kills) an electron
in orbital $q$.  The orbital indices can be uniquely mapped to the relevant orbital quantum numbers ($n_x, n_y, n_z$ for
the PIW NC model) so that the orbital energies can be readily computed (see SI for more details). 

\begin{figure}
\begin{center}
\includegraphics[width=6in]{figs/Au_AllThree_Alternate.png}
\caption{Three regimes for light-matter interactions leading to unique
spatial and temporal shaping of the incident field, and the corresponding
impact on electronic dynamics in a Au nanocube. Plots of the near-field enhancements (${\bf |E|}/{\bf |E_0|}$) are shown for the
Au NC LSPR ($\lambda=532 nm$, {\bf Panel (a)}), a Fabry-Perot resonance of a d=270nm dielectric nanosphere decorated with Au NCs ($\lambda = 397 nm$, {\bf Panel (b)}),
and a Whispering Gallery Mode resonance of a d=685 nm dielectric nanosphere decorated with Au NCs ($\lambda = 493 nm$, {\bf Panel (c)}).
The extinction spectra of these three structures are shown overlaid with the dipole-allowed transitions in the PIW model of the Au NC, showing particularly
strong overlap between these transitions and the scattering resonances of the $d=685 nm$ dielectric nanosphere ({\bf Panel (d)}).
The change in orbital populations ($D_p^p(t)-D_P^p(t=0)$) is computed to measure hot-electron and hot-hole generation.
Both dielectric scattering resonances show more efficient generation of hot-electrons ({\bf Panel (e)}) and hot-holes ({\bf Panel(f)}) compared to LSPR in this case.}
\end{center}
\end{figure}

%% E_f gold is 5.52 eV, Nels = 472
%% E_f Pt   is 9.40 eV, Nels = 1104
The TDCIS approach is applied to investigate hot carrier dynamics of $L=2nm$ gold and platinum NCs that are (a) free-standing so that
the dynamics are induced by optical resonances supported by the NC alone and (b) supported on various dielectric nanospheres so
that the dynamics are also induced by time-evolving nearfields arising from dielectric scattering resonance.  We choose these metals because 
attachment of gold and platinum metal nanoparticles to various sized dielectric nanospheres has been experimentally demonstrated; future
work will investigate similar phenomena with non-precious metals.
The electronic structure
of these model NCs are distinguishable by their Fermi energies and the number of electrons: the Au NC model has a Fermi energy
of 5.52 eV (compared to the bulk value of 5.53 eV) and 472 electrons, the Pt NC model has a Fermi energy of 9.40 eV (compared to the bulk
value of 9.75 eV) and 1104 electrons. 
In this work, 4900 singly-excited configurations are included in $|\Psi_{CIS}\rangle$ for both Au and Pt NCs.  
Despite the large number of excited states included in our many-electron wavefunctions, the high 
degeneracy of the underlying dipole-allowed transitions leads to a relatively small number of visible spectroscopic lines for the Au (see Figure 1 (d))
and Pt models (see Figure 2 (d)). 

The spectral flexibility of dielectric scattering
resonances allows tuning to overlap with one or more of these transition energies via the nanosphere size; in contrast, for very small 
metal nanoparticles, the position of the LSPR is intrinsically related to the relative permittivity of the metal~\cite{Bohren}.  This is
illustrated by plotting the scattering spectrum of a moderate-sized ($d=270nm$) and large ($d=685nm$) dielectric nanospheres and the 
absorption spectrum of a small ($d=2nm$) gold nanosphere all computed via Mie theory~\cite{Bohren}.  These Mie spectra are overlaid with the dipole-allowed
transitions in our PIW model of the Au NC (see Figure 1(d)). The $d=270nm$ dielectric nanosphere has a relatively broad scattering resonance 
(herein referred to as a Fabry-Perot [FP] resonance) that overlaps 
with a PIW Au NC transition at 3.1 eV, and
a $d=685nm$ dielectric nanosphere has narrow scattering resonances (whispering gallery modes [WGM]) 
that overlap with multiple transitions between 1.3 and 3.3 eV.  The $2nm$ Au nanosphere has a broad absorption peak associated with its LSPR that partially
overlaps with a dipole-allowed transition in the PIW Au NC at 2.5 eV. 

As proxies of the hot carrier dynamics in the Au NC, we plot the population dynamics of the highest and lowest energy orbitals in the Au NC active space, which lie
2.2 eV above and 1.6 eV below the Fermi energy, respectively (Figure 1 (e) and (f)).  Snapshots of the populations of all active
orbitals in the Au NC model at various times are also provided in the SI (see Figure S1).  
We observe that the LSPR generates a relatively small
population of hot carriers with field-driven dynamics that evolve on a short (~10 fs) timescale.  FP resonance in this
case leads to significantly more efficient hot-hole generation with field driven dynamics that evolve on a moderate (~50 fs) timescale (Figure 1(f)), while
WGM resonances lead to significantly more efficient hot-electron generation with field drive dynamics that evolve on a much longer (~200 fs) timescale
(Figure 1(e) and Figure 3(b)).  Another key distinction between the hot carrier density generated by the Au LSPR and by SMA is that the hot-carrier density
of the former is concentrated within 1 eV of the Fermi level, while SMA leads to appreciable density $\pm 2$ eV of the Fermi level (see Figure S1).

These differences can be attributed in part to the unique ways in which each resonant interaction simultaneously modulates the incident optical
fields in space and in time. An illustration of spatial confinement associated with each resonance can be seen in the
electric field intensity maps at the resonance frequency for the
Au LSPR (Figure 1(a)), FP resonance (Figure 1(b)), and WGM resonance (Figure 1(c)); all resonances lead to approximately 1 order of
magnitude nearfield enhancement in the vicinity of the Au NC.
A progression of resonance lifetimes can be inferred from the extinction spectra of the Au LSPR, the FP resonance, and the WGM resonance, with
the Au LSPR having the broadest extinction peak and the shortest lifetime, and the WGM having the most narrow extinction peaks and longest lifetimes.  These
lifetimes determine the period of time that the metal electrons are being driven by enhanced nearfields associated with the optical resonances, and
the long lifetime associated with the WGM means that the nearfield that drives the metal electrons continues to evolve on a relatively long timescale compared
to the electronic dynamics.
Longevity of the evolution of the electric field resulting from dielectric scattering resonances is consistent with the 
longer period of population transfer from the Fermi level and below to higher unoccupied orbitals when compared with surface plasmon
resonance of Au.  The differences in these dynamics can be clearly seen in the time traces of the populations driven
by the LSRP and FP resonance, which are virtually flat after ~20 fs, compared to the populations driven by WGM resonances, which continue to grow over
a period of ~100 fs (see Figure 2 (e) and (f)). 

\begin{figure}
\begin{center}
\includegraphics[width=6in]{figs/Pt_AllThree_Alternate.png}
\caption{Three regimes for light-matter interactions leading to unique
spatial and temporal shaping of the incident field, and the corresponding
impact on electronic dynamics in a Pt nanocube. Plots of the near-field enhancements (${\bf |E|}/{\bf |E_0|}$) are shown of the
Pt NC's extinction maximum ($\lambda=200 nm$, {\bf Panel (a)}), a Fabry-Perot resonance of a d=270nm dielectric nanosphere decorated with Pt NCs
($\lambda = 397 nm$, {\bf Panel (b)}),
and a Whispering Gallery Mode resonance of a d=685 nm dielectric nanosphere decorated with Pt NCs ($\lambda = 493 nm$, {\bf Panel (c)}).
The extinction spectra of these three structures are shown overlaid with the dipole-allowed transitions in the PIW model of the Pt NC, showing partial
overlap between these transitions and the scattering resonances of the $d=685 nm$ dielectric nanosphere ({\bf Panel (d)}).
The change in orbital populations ($D_p^p(t)-D_P^p(t=0)$) is computed to measure hot-electron and hot-hole generation.
The WGM shows much more efficient generation of hot-electrons ({\bf Panel (e)}) and hot-holes ({\bf Panel(f)}) compared to FP resonance in this case.}
\end{center}
\end{figure}

In contrast to Au, small Pt nanoparticles do not support LSPR at visible wavelengths (see Figure 2 (d)), and consequently, we do not
observe any overlap between an absorption resonance of the $2nm$ Pt nanoparticle and the dipole-allowed 
transitions of our PIW Pt NC model.  We also consider
the same geometries of the dielectric nanospheres as before ($d=270nm$ and $d=685 nm$); the scattering resonances of the former have partial overlap with transitions in the PIW Pt NC model at 1.6 and 1.8 eV, and the resonances of the latter have partial 
overlap with transitions at 1.8, 1.8, and 1.95 eV.  Again, the dielectric scattering resonances provide 
approximately an order of magnitude 
nearfield enhancement in the vicinity of the Pt NC, while the nearfield enhancement provided by the lone 
Pt nanoparticle is much smaller ($|{\bf E}|/|{\bf E_0}| \approx 2$ at $\lambda=200nm$ corresponding to the extinction maximum, see Figure 2 (a), (b), and (c)).
Due to its relatively weak optical interaction, scattering of the Pt nanoparticle does not lead to a large density of hot carriers; small populations
of carriers are created near the Fermi energy (see Figure S2).  These hot carriers can be attributed mostly to direct interaction with the 
incident pulse, which has a peak field strength of $E_0 \approx 614,000,000 \: V/m$, (see SI for more details).
SMA on the $d=270nm$ nanosphere
leads to generation of hot carriers within 1 eV of the Fermi energy with dynamics that persist for moderate (~50 fs) timescales.
The WGMs supported by the $d=685$ dielectric NS show much greater efficiency in creating energetic carriers compared
to the resonances of the $d=270nm$ dielectric NS or the Pt nanoparticle.  For example, WGMs supported by the $d=685nm$ nanosphere are three times
more effective at generating holes in the lowest lying orbital ($E_F - 1.6 eV$) 
and electrons in the highest 
lying orbital ($E_F + 0.84 eV$) in the PIW Pt NC active space as compared to FP resonance of the $d=270nm$ NS (see Figure 2 (e) and (f)).  
The hot carriers driven by WGM also show much more persistent dynamics, and continue to evolve beyond 100 fs (see Figure 2 (e) and (f) and Figure S2).
\begin{figure}
\begin{center}
\includegraphics[width=6in]{figs/Au_WGM_Spectrum_and_Trajectories.png}
\caption{Fine-shaping of the spatial and temporal profile of the incident field
through the geometry of the dielectric nanosphere.  Panel {\bf (a)} shows
the scattering spectra of a size progression of dielectric nanospheres that display
predominately Whispering Gallery Mode visible resonances.  The
$685nm$ nanosphere's scattering spectra has the best overlap with the dipole-allowed
transitions in the PIW Au NC model, and this structure shows most efficient
generation of hot-electrons in the most energetic orbitals included in our
active space (Panel {\bf (b)}), and more efficiently generation of
hot holes in the lowest energy orbitals included in the Au NC active space (Panel
{\bf (c)}).   }
\end{center}
\end{figure}

To demonstrate the exquisite tunability of the dielectric scattering resonances, and their 
commensurate impact on the electronic dynamics, we consider SMA by a series of 5 different Au-decorated dielectric
nanospheres.  This progression from $d=705 nm$ to $d=685 nm$ brings the dielectric scattering modes progressively 
into resonance with the dipole-allowed transition in the PIW Au NC (see Figure 3 (a)).  We observe efficient hot carrier
generation from SMA in all of these structures, though SMA from the $d=685 nm$ structure leads to most efficient generation
of hot carriers in the highest and lowest energy orbitals in the Au NC active space (see Figure 3(b) and 3(c)), 
most likely owing to the strong overlap between the scattering resonances and the transitions in the Au NC.


%In this work, we develop
%a multi-scale modeling approach for elucidating the hot carrier dynamics initiated by scattering mediated
%absorption.  Our calculations reveal that unique hot carrier distributions and dynamics arise
%from scattering mediated absorption as compared plasmon excitation, and also suggest that in
%a variety of circumstances, scattering mediated absorption may lead to more
%efficient hot carrier generation than plasmon resonance under the same external illumination
%conditions.  These results are an important first step in understanding the phenomena of
%scattering mediate hot carrier generation, which has potential for expanding the
%palette of materials that can be utilized for hot carrier mediated photochemistry beyond plasmonic metals,
%and for enabling unique pathways for photocatalytic transformations.

In this work, we have developed a multi-scale theoretical approach that utilizes time-domain electrodynamics to achieve a rigorous description of how light is shaped in space and time by interactions with complex nanostructured matter, and bridges this information with a real time-dependent configuration interaction singles approach to electronic dynamics.  Importantly, this method goes beyond a perturbative inclusion of the electric field and enables us to follow the electronic degrees of freedom subject to strong fields with arbitrary time-dependence, both of which are characteristic of the nearfields resulting from the nanophotonics resonances explored here.  The explicit inclusion of time-domain fields would also enable investigation of time-resolved spectroscopy experiments on similar structures, where the incident fields themselves may be shaped in space and time.  

We have applied this methodology to the study hot carrier generation in novel hybrid nanostructures consisting of large dielectric core structures decorated by metal nanoparticles that enable the interplay between scattering resonances in the core structure and broadband absorption in the metal nanoparticles, a phenomena known as scattering mediated absorption.  We have shown that scattering mediated absorption can generate hot carriers while completely circumventing plasmonic resonance.  This result points to the potential for realizing efficient hot carrier generation and transfer in non-precious metals that are earth abundant and cost-effective; a possibility that will be explored in future work.   Our results also reveal that unique electronic dynamics can be realized by modulating the frequency and linewidth of the scattering resonances, which can be controlled by the size and/or refractive index of the dielectric structure.  In particular, high quality factor scattering resonances (e.g. whispering galler modes) can induce electronic dynamics that persist for hundreds of femotoseconds, while plasmonic resonance induces field-driven dynamics on a timescale of ~10 fs.  The ability to drive excited-state dynamics of hot carriers through tuning of these resonances may have important implications for improving the efficiency and selectivity of photocatalysis via hot carrier transfer, which can be severely limited by the rapid decay of energetic carriers resulting from plasmon resonances.

\newpage
\section{Supplementary Information}

\subsection{Plots of Global Hot-Carrier Distributions}

\begin{figure}
\begin{center}
\includegraphics[width=6in]{figs/Au_HotElectronDistribution_Comparison.png}
\caption{Snapshots of changes in occupation of each orbital in the active space of the PIW Au NC model as a measure of instantaneous hot carrier distributions.
The change in orbital occupation is computed from elements of the time-dependent one-electron reduced density matrix ($^1$RDM) relative to
their initial value, $D_p^p(t)-D_P^p(t=0)$ for several timepoints in the simulation.}
\end{center}
\end{figure}


\begin{figure}
\begin{center}
\includegraphics[width=6in]{figs/Pt_HotElectronDistribution_Comparison.png}
\caption{Snapshots of changes in occupation of each orbital in the active space of the PIW Pt NC model as a measure of instantaneous hot carrier distributions.
The change in orbital occupation is computed from elements of the time-dependent one-electron reduced density matrix ($^1$RDM) relative to
their initial value, $D_p^p(t)-D_P^p(t=0)$ for several timepoints in the simulation.  }
\end{center}
\end{figure}



\subsection{Electronic structure of metal nanocubes}
For cubic metal nanoparticles, we approximate the one-electron orbitals as energy eigenstates of the particle-in-a-cubic-well.  
For a particle confined by a cubic well with length $L$, the potential is 0 when $x<L, y<L, z<L$ and infinity otherwise.  The energy eigenstates
have the form
\begin{equation}
\psi_{nx,ny,nz} = \left(\frac{2}{L}\right)^{3/2} \: {\rm sin}\left(\frac{n_x \: \pi \: x}{L}\right) {\rm sin}\left(\frac{n_y \: \pi \: y}{L}\right) {\rm sin}\left(\frac{n_z \: \pi \: z}{L}\right).
\end{equation}
The energy eigenvalues have the form
\begin{equation}
\epsilon_{nx,ny,nz} = \frac{\hbar^2 \pi^2}{2 \: m \: L^2}\left(n_x^2 + n_y^2 + n_z^2\right).
\end{equation}
The transition dipole operator can be decomposed into its components,
\begin{equation}
{\bf \hat{\mu} } = \hat{\mu}_x \: {\bf i} + \hat{\mu}_y \: {\bf j} + \hat{\mu}_z \: {\bf k}.
\end{equation}
The transition dipole integral components can be evaluated analytically, for example, the 
$x$-component has the form
\begin{align*}
\langle \psi_{nx,ny,nz} |  \hat{\mu}_x | \psi_{nx',ny',nz'} \rangle = e \: \delta_{ny,ny'} \: \delta_{nz,nz'} \:
\frac{L (\pi (n_x - n_x'){\rm sin}(\pi(n_x - n_x'))+{\rm cos}(\pi(n_x-n_x'))-1) }{\pi^2 (n_x - n_x')^2 } \\
-  e \: \delta_{ny,ny'} \: \delta_{nz,nz'} \:
\frac{L (\pi (n_x + n_x'){\rm sin}(\pi(n_x + n_x'))+{\rm cos}(\pi(n_x+n_x'))-1) }{\pi^2 (n_x + n_x')^2 },
\end{align*}
where $\hat{\mu}_x = -e x$.  Analogous expressions can be obtained for expectation values of $\hat{\mu}_y$ and $\hat{\mu}_z$. 

We order the orbitals by a single index $p$ such that $\epsilon_{p+1} \geq \epsilon_p$; that is,
each $\psi_{nx,ny,nz}$ can be uniquely labeled $\psi_p$.
Using the above expressions and following this labeling scheme, the diagonal matrix elements can be evaluated as
\begin{align}
\langle \Phi_0 | \hat{H}(t) | \Phi_0 \rangle &= \sum_{p=1}^{nocc} \epsilon_p \\
\langle \Phi_i^a | \hat{H}(t) | \Phi_i^a \rangle &= \sum_{p=1}^{nocc} \epsilon_p - \epsilon_i + \epsilon_a
\end{align}
and the off-diagonal matrix elements can be evaluated as
\begin{align}
\langle \Phi_0 | \hat{H}(t) | \Phi_i^a \rangle &=  {\bf E}(t) \cdot \langle \psi_i |  {\bf \hat{\mu}} | \psi_a \rangle \\
\langle \Phi_i^a | \hat{H}(t) | \Phi_j^b \rangle &=   {\bf E}(t) \cdot \langle \psi_a |  {\bf \hat{\mu}} | \psi_b \rangle \delta_{ij}  - {\bf E}(t) \cdot \langle \psi_i | {\bf \hat{\mu}} | \psi_j \rangle \delta_{ab}.
\end{align} 



\subsection{Finite-difference time-domain calculations}
A commercial simulator based on the finite-difference time-domain method~\cite{Lumerical} was used to compute the electric field, $E(t)$
1 \AA $\:$  
away from the nanoparticle surface in each of the scenarios considered.  The displacement
was taken along the $z$-axis, corresponding to the polarization direction of incident light since the strongest
near-field enhancement is expected along this direction.  A grid spacing of 1 \AA $\:$  
in $x$, $y$, and $z$ was utilized
in a cubic region extending 1 nm beyond the metal NP surface, and a non-uniform mesh was utilized otherwise with $dx$, $dy$, $dz \leq 20 nm$.
For each composite structure, a nanoparticle was placed at the surface of the dielectric nanosphere at an angle of
20$^{\circ}$ with respect to the propagation axis of the incident light. In all simulations, light propagates
along the $x$ axis and is polarized along the $z$ axis.  The metal nanoparticles are centered at $y=0$.  
A total-field scattered-field source was used to illuminate the structures.  The FDTD simulations were terminated when the 
ratio of the total energy in the simulation volume to the total energy injected by the illumination source falls below
$10^{-6}$.  Because the WGMs are higher quality factor resonances, longer time is typically required for these simulations
as compared to the plasmonic particles alone.  

The resulting time-domain fields were fed into our TDCIS algorithm, allowing us to simulate the electronic dynamics
driven by rigorously-computed nearfields from scattering and plasmon resonances, which show strong spatiotemporal modification relative
to freely propagating light.  The electric field was scaled by a factor $E_0 \approx 614,000,000 \: V/m$ so that the peak power
of the illumination source is $10^{15} \: W/m^2$.  The electric field was sampled at intervals of approximately 2.8 attoseconds for all simulations, which leads
to a time-step that ensures stability of
the wavefunction propagation with the relevant energy scales of our simulations.  Our TDCIS scheme
requires the evaluation of the electric field at intermediate times between these timesteps, and we use a simple update
based on centered-finite differences to approximate the electric fields at these times.  As an example, if the 
electric field is known at times $t_1$, $t_2 = t_1 + dt$, and $t_3 = t_1 + 2\cdot dt$ where $dt = 2.8 \: as$, and knowledge
of the field is required at some time $t_m = t_2 + m\cdot dt$ where $m$ is non-integer, $E(t_m)$ is estimated as follows: 
\begin{equation}
{\bf E}(t_m) =  {\bf E}(t_2) + \frac{{\bf E}(t_3)-{\bf E}(t_1) }{t_3 - t_1 } \cdot m\cdot dt.
\end{equation}

The optical response of Au and Pt in the FDTD simulations utilizes permittivity data from the work of Johnson and Christy~\cite{JC_PRB_1972} and Palik~\cite{Palik}, respectively.  We assume a static dielectric constant of 2.6 for
the dielectric nanospheres in this work, which is comparable to the visible dielectric constant of titanium dioxide. 

\section{Acknowledgment}
This work was performed, in part, utilizing resources at
the Center for Nanoscale Materials, a US Department of Energy, Office of Science, Office of
Basic Energy Sciences User Facility (contract no. DE-AC02-06CH11357).
JJF Acknowledges the College of Science and Health for startup support.
J.C. and N.E. acknowledge the NSF-GS-LSAMP for support.  K.F. acknowledges the WPU CFR for support.
$^{\dagger}$J.C., N.E., and K.F. contributed equally to this work.
\bibliography{SMHET} 

\end{document}
   

